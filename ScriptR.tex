% Options for packages loaded elsewhere
\PassOptionsToPackage{unicode}{hyperref}
\PassOptionsToPackage{hyphens}{url}
%
\documentclass[
]{article}
\usepackage{amsmath,amssymb}
\usepackage{iftex}
\ifPDFTeX
  \usepackage[T1]{fontenc}
  \usepackage[utf8]{inputenc}
  \usepackage{textcomp} % provide euro and other symbols
\else % if luatex or xetex
  \usepackage{unicode-math} % this also loads fontspec
  \defaultfontfeatures{Scale=MatchLowercase}
  \defaultfontfeatures[\rmfamily]{Ligatures=TeX,Scale=1}
\fi
\usepackage{lmodern}
\ifPDFTeX\else
  % xetex/luatex font selection
\fi
% Use upquote if available, for straight quotes in verbatim environments
\IfFileExists{upquote.sty}{\usepackage{upquote}}{}
\IfFileExists{microtype.sty}{% use microtype if available
  \usepackage[]{microtype}
  \UseMicrotypeSet[protrusion]{basicmath} % disable protrusion for tt fonts
}{}
\makeatletter
\@ifundefined{KOMAClassName}{% if non-KOMA class
  \IfFileExists{parskip.sty}{%
    \usepackage{parskip}
  }{% else
    \setlength{\parindent}{0pt}
    \setlength{\parskip}{6pt plus 2pt minus 1pt}}
}{% if KOMA class
  \KOMAoptions{parskip=half}}
\makeatother
\usepackage{xcolor}
\usepackage[margin=1in]{geometry}
\usepackage{color}
\usepackage{fancyvrb}
\newcommand{\VerbBar}{|}
\newcommand{\VERB}{\Verb[commandchars=\\\{\}]}
\DefineVerbatimEnvironment{Highlighting}{Verbatim}{commandchars=\\\{\}}
% Add ',fontsize=\small' for more characters per line
\usepackage{framed}
\definecolor{shadecolor}{RGB}{248,248,248}
\newenvironment{Shaded}{\begin{snugshade}}{\end{snugshade}}
\newcommand{\AlertTok}[1]{\textcolor[rgb]{0.94,0.16,0.16}{#1}}
\newcommand{\AnnotationTok}[1]{\textcolor[rgb]{0.56,0.35,0.01}{\textbf{\textit{#1}}}}
\newcommand{\AttributeTok}[1]{\textcolor[rgb]{0.13,0.29,0.53}{#1}}
\newcommand{\BaseNTok}[1]{\textcolor[rgb]{0.00,0.00,0.81}{#1}}
\newcommand{\BuiltInTok}[1]{#1}
\newcommand{\CharTok}[1]{\textcolor[rgb]{0.31,0.60,0.02}{#1}}
\newcommand{\CommentTok}[1]{\textcolor[rgb]{0.56,0.35,0.01}{\textit{#1}}}
\newcommand{\CommentVarTok}[1]{\textcolor[rgb]{0.56,0.35,0.01}{\textbf{\textit{#1}}}}
\newcommand{\ConstantTok}[1]{\textcolor[rgb]{0.56,0.35,0.01}{#1}}
\newcommand{\ControlFlowTok}[1]{\textcolor[rgb]{0.13,0.29,0.53}{\textbf{#1}}}
\newcommand{\DataTypeTok}[1]{\textcolor[rgb]{0.13,0.29,0.53}{#1}}
\newcommand{\DecValTok}[1]{\textcolor[rgb]{0.00,0.00,0.81}{#1}}
\newcommand{\DocumentationTok}[1]{\textcolor[rgb]{0.56,0.35,0.01}{\textbf{\textit{#1}}}}
\newcommand{\ErrorTok}[1]{\textcolor[rgb]{0.64,0.00,0.00}{\textbf{#1}}}
\newcommand{\ExtensionTok}[1]{#1}
\newcommand{\FloatTok}[1]{\textcolor[rgb]{0.00,0.00,0.81}{#1}}
\newcommand{\FunctionTok}[1]{\textcolor[rgb]{0.13,0.29,0.53}{\textbf{#1}}}
\newcommand{\ImportTok}[1]{#1}
\newcommand{\InformationTok}[1]{\textcolor[rgb]{0.56,0.35,0.01}{\textbf{\textit{#1}}}}
\newcommand{\KeywordTok}[1]{\textcolor[rgb]{0.13,0.29,0.53}{\textbf{#1}}}
\newcommand{\NormalTok}[1]{#1}
\newcommand{\OperatorTok}[1]{\textcolor[rgb]{0.81,0.36,0.00}{\textbf{#1}}}
\newcommand{\OtherTok}[1]{\textcolor[rgb]{0.56,0.35,0.01}{#1}}
\newcommand{\PreprocessorTok}[1]{\textcolor[rgb]{0.56,0.35,0.01}{\textit{#1}}}
\newcommand{\RegionMarkerTok}[1]{#1}
\newcommand{\SpecialCharTok}[1]{\textcolor[rgb]{0.81,0.36,0.00}{\textbf{#1}}}
\newcommand{\SpecialStringTok}[1]{\textcolor[rgb]{0.31,0.60,0.02}{#1}}
\newcommand{\StringTok}[1]{\textcolor[rgb]{0.31,0.60,0.02}{#1}}
\newcommand{\VariableTok}[1]{\textcolor[rgb]{0.00,0.00,0.00}{#1}}
\newcommand{\VerbatimStringTok}[1]{\textcolor[rgb]{0.31,0.60,0.02}{#1}}
\newcommand{\WarningTok}[1]{\textcolor[rgb]{0.56,0.35,0.01}{\textbf{\textit{#1}}}}
\usepackage{graphicx}
\makeatletter
\def\maxwidth{\ifdim\Gin@nat@width>\linewidth\linewidth\else\Gin@nat@width\fi}
\def\maxheight{\ifdim\Gin@nat@height>\textheight\textheight\else\Gin@nat@height\fi}
\makeatother
% Scale images if necessary, so that they will not overflow the page
% margins by default, and it is still possible to overwrite the defaults
% using explicit options in \includegraphics[width, height, ...]{}
\setkeys{Gin}{width=\maxwidth,height=\maxheight,keepaspectratio}
% Set default figure placement to htbp
\makeatletter
\def\fps@figure{htbp}
\makeatother
\setlength{\emergencystretch}{3em} % prevent overfull lines
\providecommand{\tightlist}{%
  \setlength{\itemsep}{0pt}\setlength{\parskip}{0pt}}
\setcounter{secnumdepth}{-\maxdimen} % remove section numbering
\ifLuaTeX
  \usepackage{selnolig}  % disable illegal ligatures
\fi
\IfFileExists{bookmark.sty}{\usepackage{bookmark}}{\usepackage{hyperref}}
\IfFileExists{xurl.sty}{\usepackage{xurl}}{} % add URL line breaks if available
\urlstyle{same}
\hypersetup{
  pdftitle={Proyecto III - Estadística Matemática},
  pdfauthor={José Donado, Ian Castellanos, Joaquín Puente, Micaela Yataz},
  hidelinks,
  pdfcreator={LaTeX via pandoc}}

\title{Proyecto III - Estadística Matemática}
\author{José Donado, Ian Castellanos, Joaquín Puente, Micaela Yataz}
\date{2025-10-08}

\begin{document}
\maketitle

\hypertarget{problema-1-17-puntos}{%
\subsubsection{Problema 1 (17 puntos)}\label{problema-1-17-puntos}}

La Agencia de Protección Ambiental en conjunto con la Universidad de
Florida recientemente realizó un amplio estudio de los posibles efectos
de trazas de elementos en el agua potable sobre la formación de cálculos
renales. La tabla siguiente presenta los datos de edad, la cantidad de
calcio en agua potable (medida en partes por millón) y los hábitos de
fumar. Estos datos se obtuvieron de individuos con problemas actuales de
cálculos renales, todos los cuales vivían en las dos Carolinas y en
estados de las Montañas Rocallosas.

\begin{Shaded}
\begin{Highlighting}[]
\CommentTok{\#aqui va la tabla}
\end{Highlighting}
\end{Shaded}

\begin{enumerate}
\def\labelenumi{\alph{enumi})}
\tightlist
\item
  Estime la concentración promedio de calcio en el agua potable para
  pacientes con cálculos renales en las Carolinas. Establezca un límite
  para el error de estimación (2 desviaciones estándar).
\end{enumerate}

Gracias a que se tiene una muestra grande y el parámetro muestral, se
puede utilizar el Teorema del Límite Central y la distribución normal
(Z) para calcular el promedio muestral (puntual) con su correspondiente
intervalo de confianza.

De primero, calcularemos el límite del Error. Este se define como el
valor crítico (Zα/2\hspace{0pt}) y el Error Estándar de la Media
(σxˉ\hspace{0pt}).

El Error Estándar de la Media se calcula como
\(\sigma_{\bar{x}} = \frac{s}{\sqrt{n}}\), donde \(s\) es la desviación
estándar muestral y \(n\) es el tamaño de la muestra.

Entonces\ldots{}

\begin{enumerate}
\def\labelenumi{\alph{enumi})}
\item
  Calcule la diferencia en edades medias para pacientes con cálculos
  renales en las Ca- rolinas y en las Rocallosas. Fije un límite para el
  error de estimación (2 desviaciones estándar).
\item
  Calcule y precise un límite de desviación estándar de 2 en la
  diferencia en proporciones de pacientes con cálculos renales, de las
  Carolinas y las Rocallosas, que eran fumadores en el momento de hacer
  el estudio
\end{enumerate}

\#\#\#Problema 2

Suponga que Y está distribuida normalmente con media \(0\) y varianza
\(\sigma^2\) desconocida. a) Encuentre la distribución de
\(Y^2/ \sigma^2\). ¿Por qué? b) Hallar un intervalo de confianza del
\(95\%\) para \(\sigma^2\). c) Encontrar un límite de confianza superior
de \(95\%\) para \(\sigma^2\). d) Hallar un límite de confianza inferior
de \(95%
\) para \(\sigma^2\).

\#\#\#Problema 3

Dos marcas de refrigeradores, denotadas por A y B, están garantizadas
por 1 año. En una muestra aleatoria de 50 refrigeradores de la marca A,
se observó que 12 de ellos fallaron antes de terminar el periodo de
garantía. Una muestra aleatoria independiente de 60 refrigeradores de la
marca B también reveló 12 fallas durante el período de garantía.

\begin{Shaded}
\begin{Highlighting}[]
\CommentTok{\# Datos base}
\NormalTok{n1 }\OtherTok{\textless{}{-}} \DecValTok{50}\NormalTok{; x1 }\OtherTok{\textless{}{-}} \DecValTok{12}   \CommentTok{\# Marca A}
\NormalTok{n2 }\OtherTok{\textless{}{-}} \DecValTok{60}\NormalTok{; x2 }\OtherTok{\textless{}{-}} \DecValTok{12}   \CommentTok{\# Marca B}

\NormalTok{alpha }\OtherTok{\textless{}{-}} \FloatTok{0.02}
\NormalTok{z\_crit }\OtherTok{\textless{}{-}} \FunctionTok{qnorm}\NormalTok{(}\DecValTok{1} \SpecialCharTok{{-}}\NormalTok{ alpha}\SpecialCharTok{/}\DecValTok{2}\NormalTok{)  }\CommentTok{\# z\_0.99 para 98\% de confianza}

\CommentTok{\# Estimadores puntuales}
\NormalTok{p1\_hat }\OtherTok{\textless{}{-}}\NormalTok{ x1}\SpecialCharTok{/}\NormalTok{n1}
\NormalTok{p2\_hat }\OtherTok{\textless{}{-}}\NormalTok{ x2}\SpecialCharTok{/}\NormalTok{n2}
\NormalTok{diff\_hat }\OtherTok{\textless{}{-}}\NormalTok{ p1\_hat }\SpecialCharTok{{-}}\NormalTok{ p2\_hat}

\CommentTok{\# Error estándar para diferencia de proporciones independientes}
\NormalTok{se }\OtherTok{\textless{}{-}} \FunctionTok{sqrt}\NormalTok{( p1\_hat}\SpecialCharTok{*}\NormalTok{(}\DecValTok{1} \SpecialCharTok{{-}}\NormalTok{ p1\_hat)}\SpecialCharTok{/}\NormalTok{n1 }\SpecialCharTok{+}\NormalTok{ p2\_hat}\SpecialCharTok{*}\NormalTok{(}\DecValTok{1} \SpecialCharTok{{-}}\NormalTok{ p2\_hat)}\SpecialCharTok{/}\NormalTok{n2 )}
\end{Highlighting}
\end{Shaded}

\begin{enumerate}
\def\labelenumi{\alph{enumi})}
\tightlist
\item
  Calcule la diferencia real \((p_1 - p_2)\) entre las proporciones de
  fallas durante el período de garantía, con un coeficiente de confianza
  de aproximadamente .98, donde p1 y p2 se usaron para denotar las
  proporciones de refrigeradores de las marcas A y B, respectivamente,
  que fallaron durante los períodos de garantía.
\end{enumerate}

\begin{Shaded}
\begin{Highlighting}[]
\NormalTok{ci\_low  }\OtherTok{\textless{}{-}}\NormalTok{ diff\_hat }\SpecialCharTok{{-}}\NormalTok{ z\_crit}\SpecialCharTok{*}\NormalTok{se}
\NormalTok{ci\_high }\OtherTok{\textless{}{-}}\NormalTok{ diff\_hat }\SpecialCharTok{+}\NormalTok{ z\_crit}\SpecialCharTok{*}\NormalTok{se}

\FunctionTok{cat}\NormalTok{(}\StringTok{"p1\_hat ="}\NormalTok{, }\FunctionTok{round}\NormalTok{(p1\_hat, }\DecValTok{4}\NormalTok{), }\StringTok{"}\SpecialCharTok{\textbackslash{}n}\StringTok{"}\NormalTok{)}
\end{Highlighting}
\end{Shaded}

\begin{verbatim}
## p1_hat = 0.24
\end{verbatim}

\begin{Shaded}
\begin{Highlighting}[]
\FunctionTok{cat}\NormalTok{(}\StringTok{"p2\_hat ="}\NormalTok{, }\FunctionTok{round}\NormalTok{(p2\_hat, }\DecValTok{4}\NormalTok{), }\StringTok{"}\SpecialCharTok{\textbackslash{}n}\StringTok{"}\NormalTok{)}
\end{Highlighting}
\end{Shaded}

\begin{verbatim}
## p2_hat = 0.2
\end{verbatim}

\begin{Shaded}
\begin{Highlighting}[]
\FunctionTok{cat}\NormalTok{(}\StringTok{"Diferencia puntual (p1 {-} p2) ="}\NormalTok{, }\FunctionTok{round}\NormalTok{(diff\_hat, }\DecValTok{4}\NormalTok{), }\StringTok{"}\SpecialCharTok{\textbackslash{}n}\StringTok{"}\NormalTok{)}
\end{Highlighting}
\end{Shaded}

\begin{verbatim}
## Diferencia puntual (p1 - p2) = 0.04
\end{verbatim}

\begin{Shaded}
\begin{Highlighting}[]
\FunctionTok{cat}\NormalTok{(}\StringTok{"z\_crit (98\%) ="}\NormalTok{, }\FunctionTok{round}\NormalTok{(z\_crit, }\DecValTok{6}\NormalTok{), }\StringTok{"}\SpecialCharTok{\textbackslash{}n}\StringTok{"}\NormalTok{)}
\end{Highlighting}
\end{Shaded}

\begin{verbatim}
## z_crit (98%) = 2.326348
\end{verbatim}

\begin{Shaded}
\begin{Highlighting}[]
\FunctionTok{cat}\NormalTok{(}\StringTok{"SE ="}\NormalTok{, }\FunctionTok{round}\NormalTok{(se, }\DecValTok{6}\NormalTok{), }\StringTok{"}\SpecialCharTok{\textbackslash{}n}\StringTok{"}\NormalTok{)}
\end{Highlighting}
\end{Shaded}

\begin{verbatim}
## SE = 0.079465
\end{verbatim}

\begin{Shaded}
\begin{Highlighting}[]
\FunctionTok{cat}\NormalTok{(}\StringTok{"IC 98\% para (p1 {-} p2): ["}\NormalTok{, }\FunctionTok{round}\NormalTok{(ci\_low, }\DecValTok{4}\NormalTok{), }\StringTok{", "}\NormalTok{, }\FunctionTok{round}\NormalTok{(ci\_high, }\DecValTok{4}\NormalTok{), }\StringTok{"]}\SpecialCharTok{\textbackslash{}n}\StringTok{"}\NormalTok{, }\AttributeTok{sep =} \StringTok{""}\NormalTok{)}
\end{Highlighting}
\end{Shaded}

\begin{verbatim}
## IC 98% para (p1 - p2): [-0.1449, 0.2249]
\end{verbatim}

\begin{enumerate}
\def\labelenumi{\alph{enumi})}
\setcounter{enumi}{1}
\tightlist
\item
  En el nivel aproximado de \(98 \%\) de confianza, ¿cuál es el mayor
  ``valor creíble'' para la diferencia en las proporciones de fallas de
  refrigeradores de las marcas A y B?
\end{enumerate}

\begin{Shaded}
\begin{Highlighting}[]
\NormalTok{upper\_98 }\OtherTok{\textless{}{-}}\NormalTok{ ci\_high}
\FunctionTok{cat}\NormalTok{(}\StringTok{"Mayor valor creíble (límite superior 98\%):"}\NormalTok{, }\FunctionTok{round}\NormalTok{(upper\_98, }\DecValTok{4}\NormalTok{), }\StringTok{"}\SpecialCharTok{\textbackslash{}n}\StringTok{"}\NormalTok{)}
\end{Highlighting}
\end{Shaded}

\begin{verbatim}
## Mayor valor creíble (límite superior 98%): 0.2249
\end{verbatim}

\begin{enumerate}
\def\labelenumi{\alph{enumi})}
\setcounter{enumi}{2}
\tightlist
\item
  En el nivel aproximado de \(98 \%\) de confianza, ¿cuál es el menor
  ``valor creíble'' para la diferencia en las proporciones de fallas de
  refrigeradores de las marcas A y B?
\end{enumerate}

\begin{Shaded}
\begin{Highlighting}[]
\NormalTok{lower\_98 }\OtherTok{\textless{}{-}}\NormalTok{ ci\_low}
\FunctionTok{cat}\NormalTok{(}\StringTok{"Menor valor creíble (límite inferior 98\%):"}\NormalTok{, }\FunctionTok{round}\NormalTok{(lower\_98, }\DecValTok{4}\NormalTok{), }\StringTok{"}\SpecialCharTok{\textbackslash{}n}\StringTok{"}\NormalTok{)}
\end{Highlighting}
\end{Shaded}

\begin{verbatim}
## Menor valor creíble (límite inferior 98%): -0.1449
\end{verbatim}

\begin{enumerate}
\def\labelenumi{\alph{enumi})}
\setcounter{enumi}{3}
\tightlist
\item
  Si \(p_1 - p_2\) es realmente igual a \(0.2251\), ¿cuál marca tiene la
  mayor proporción de fallas durante el período de garantía? ¿Qué tanto
  más grande?
\end{enumerate}

\begin{Shaded}
\begin{Highlighting}[]
\NormalTok{hyp\_diff\_d }\OtherTok{\textless{}{-}} \FloatTok{0.2251}
\NormalTok{marca\_mayor\_d }\OtherTok{\textless{}{-}} \ControlFlowTok{if}\NormalTok{ (hyp\_diff\_d }\SpecialCharTok{\textgreater{}} \DecValTok{0}\NormalTok{) }\StringTok{"A"} \ControlFlowTok{else} \ControlFlowTok{if}\NormalTok{ (hyp\_diff\_d }\SpecialCharTok{\textless{}} \DecValTok{0}\NormalTok{) }\StringTok{"B"} \SpecialCharTok{:} \StringTok{"Iguales"}
\NormalTok{cuanto\_mas\_d }\OtherTok{\textless{}{-}} \FunctionTok{abs}\NormalTok{(hyp\_diff\_d)}

\FunctionTok{cat}\NormalTok{(}\StringTok{"Supuesto p1 {-} p2 ="}\NormalTok{, hyp\_diff\_d, }\StringTok{"}\SpecialCharTok{\textbackslash{}n}\StringTok{"}\NormalTok{)}
\end{Highlighting}
\end{Shaded}

\begin{verbatim}
## Supuesto p1 - p2 = 0.2251
\end{verbatim}

\begin{Shaded}
\begin{Highlighting}[]
\FunctionTok{cat}\NormalTok{(}\StringTok{"Marca con mayor proporción de fallas:"}\NormalTok{, marca\_mayor\_d, }\StringTok{"}\SpecialCharTok{\textbackslash{}n}\StringTok{"}\NormalTok{)}
\end{Highlighting}
\end{Shaded}

\begin{verbatim}
## Marca con mayor proporción de fallas: A
\end{verbatim}

\begin{Shaded}
\begin{Highlighting}[]
\FunctionTok{cat}\NormalTok{(}\StringTok{"Diferencia absoluta:"}\NormalTok{, }\FunctionTok{round}\NormalTok{(cuanto\_mas\_d, }\DecValTok{4}\NormalTok{), }\StringTok{" ("}\NormalTok{, }\FunctionTok{round}\NormalTok{(}\DecValTok{100}\SpecialCharTok{*}\NormalTok{cuanto\_mas\_d, }\DecValTok{2}\NormalTok{), }\StringTok{"\% puntos)}\SpecialCharTok{\textbackslash{}n}\StringTok{"}\NormalTok{, }\AttributeTok{sep =} \StringTok{""}\NormalTok{)}
\end{Highlighting}
\end{Shaded}

\begin{verbatim}
## Diferencia absoluta:0.2251 (22.51% puntos)
\end{verbatim}

\begin{enumerate}
\def\labelenumi{\alph{enumi})}
\setcounter{enumi}{4}
\tightlist
\item
  Si \(p_1 - p_2\) es realmente igual a \(-0.1451\), ¿cuál marca tiene
  la mayor proporción de fallas durante el período de garantía? ¿Qué
  tanto más grande?
\end{enumerate}

\begin{Shaded}
\begin{Highlighting}[]
\NormalTok{hyp\_diff\_e }\OtherTok{\textless{}{-}} \SpecialCharTok{{-}}\FloatTok{0.1451}
\NormalTok{marca\_mayor\_e }\OtherTok{\textless{}{-}} \ControlFlowTok{if}\NormalTok{ (hyp\_diff\_e }\SpecialCharTok{\textgreater{}} \DecValTok{0}\NormalTok{) }\StringTok{"A"} \ControlFlowTok{else} \ControlFlowTok{if}\NormalTok{ (hyp\_diff\_e }\SpecialCharTok{\textless{}} \DecValTok{0}\NormalTok{) }\StringTok{"B"} \ControlFlowTok{else} \StringTok{"Iguales"}
\NormalTok{cuanto\_mas\_e }\OtherTok{\textless{}{-}} \FunctionTok{abs}\NormalTok{(hyp\_diff\_e)}

\FunctionTok{cat}\NormalTok{(}\StringTok{"Supuesto p1 {-} p2 ="}\NormalTok{, hyp\_diff\_e, }\StringTok{"}\SpecialCharTok{\textbackslash{}n}\StringTok{"}\NormalTok{)}
\end{Highlighting}
\end{Shaded}

\begin{verbatim}
## Supuesto p1 - p2 = -0.1451
\end{verbatim}

\begin{Shaded}
\begin{Highlighting}[]
\FunctionTok{cat}\NormalTok{(}\StringTok{"Marca con mayor proporción de fallas:"}\NormalTok{, marca\_mayor\_e, }\StringTok{"}\SpecialCharTok{\textbackslash{}n}\StringTok{"}\NormalTok{)}
\end{Highlighting}
\end{Shaded}

\begin{verbatim}
## Marca con mayor proporción de fallas: B
\end{verbatim}

\begin{Shaded}
\begin{Highlighting}[]
\FunctionTok{cat}\NormalTok{(}\StringTok{"Diferencia absoluta:"}\NormalTok{, }\FunctionTok{round}\NormalTok{(cuanto\_mas\_e, }\DecValTok{4}\NormalTok{), }\StringTok{" ("}\NormalTok{, }\FunctionTok{round}\NormalTok{(}\DecValTok{100}\SpecialCharTok{*}\NormalTok{cuanto\_mas\_e, }\DecValTok{2}\NormalTok{), }\StringTok{"\% puntos)}\SpecialCharTok{\textbackslash{}n}\StringTok{"}\NormalTok{, }\AttributeTok{sep =} \StringTok{""}\NormalTok{)}
\end{Highlighting}
\end{Shaded}

\begin{verbatim}
## Diferencia absoluta:0.1451 (14.51% puntos)
\end{verbatim}

\begin{enumerate}
\def\labelenumi{\alph{enumi})}
\setcounter{enumi}{5}
\tightlist
\item
  Como puede observar cero es un valor creíble de la diferencia.
  ¿Concluiría usted que hay evidencia de una diferencia en las
  proporciones de fallas (dentro del período de garantía) para las dos
  marcas de refrigeradores? ¿Por qué?
\end{enumerate}

\begin{Shaded}
\begin{Highlighting}[]
\NormalTok{incluye\_cero }\OtherTok{\textless{}{-}}\NormalTok{ (ci\_low }\SpecialCharTok{\textless{}=} \DecValTok{0}\NormalTok{) }\SpecialCharTok{\&\&}\NormalTok{ (}\DecValTok{0} \SpecialCharTok{\textless{}=}\NormalTok{ ci\_high)}
\FunctionTok{cat}\NormalTok{(}\StringTok{"IC 98\%:"}\NormalTok{, }\StringTok{"["}\NormalTok{, }\FunctionTok{round}\NormalTok{(ci\_low, }\DecValTok{4}\NormalTok{), }\StringTok{", "}\NormalTok{, }\FunctionTok{round}\NormalTok{(ci\_high, }\DecValTok{4}\NormalTok{), }\StringTok{"]}\SpecialCharTok{\textbackslash{}n}\StringTok{"}\NormalTok{, }\AttributeTok{sep =} \StringTok{""}\NormalTok{)}
\end{Highlighting}
\end{Shaded}

\begin{verbatim}
## IC 98%:[-0.1449, 0.2249]
\end{verbatim}

\begin{Shaded}
\begin{Highlighting}[]
\FunctionTok{cat}\NormalTok{(}\StringTok{"¿El IC incluye 0?:"}\NormalTok{, incluye\_cero, }\StringTok{"}\SpecialCharTok{\textbackslash{}n}\StringTok{"}\NormalTok{)}
\end{Highlighting}
\end{Shaded}

\begin{verbatim}
## ¿El IC incluye 0?: TRUE
\end{verbatim}

\begin{Shaded}
\begin{Highlighting}[]
\NormalTok{conclusion }\OtherTok{\textless{}{-}} \ControlFlowTok{if}\NormalTok{ (incluye\_cero) }\StringTok{"No hay evidencia suficiente de diferencia al 98\%."} \ControlFlowTok{else} \StringTok{"Sí hay evidencia de diferencia al 98\%."}
\FunctionTok{cat}\NormalTok{(}\StringTok{"Conclusión:"}\NormalTok{, conclusion, }\StringTok{"}\SpecialCharTok{\textbackslash{}n}\StringTok{"}\NormalTok{)}
\end{Highlighting}
\end{Shaded}

\begin{verbatim}
## Conclusión: No hay evidencia suficiente de diferencia al 98%.
\end{verbatim}

\#\#\#Problema 4

Un servicio estatal de fauna silvestre desea calcular el número promedio
de días que cada cazador con licencia se dedica a esta actividad
realmente durante una estación determinada, con un límite en el error de
estimación igual a \(2\) días de caza. Si los datos recolectados en
estudios anteriores han demostrado que s es aproximadamente igual a 10,
¿cuántos cazadores deben estar incluidos en el estudio?

\begin{Shaded}
\begin{Highlighting}[]
\CommentTok{\# Función para calcular tamaño de muestra}
\NormalTok{calcular\_n }\OtherTok{\textless{}{-}} \ControlFlowTok{function}\NormalTok{(sigma, B, }\AttributeTok{confianza =} \FloatTok{0.95}\NormalTok{) \{}
\NormalTok{  alpha }\OtherTok{\textless{}{-}} \DecValTok{1} \SpecialCharTok{{-}}\NormalTok{ confianza}
\NormalTok{  z }\OtherTok{\textless{}{-}} \FunctionTok{qnorm}\NormalTok{(}\DecValTok{1} \SpecialCharTok{{-}}\NormalTok{ alpha}\SpecialCharTok{/}\DecValTok{2}\NormalTok{)}
\NormalTok{  n }\OtherTok{\textless{}{-}}\NormalTok{ (z }\SpecialCharTok{*}\NormalTok{ sigma }\SpecialCharTok{/}\NormalTok{ B)}\SpecialCharTok{\^{}}\DecValTok{2}
  \FunctionTok{return}\NormalTok{(}\FunctionTok{ceiling}\NormalTok{(n))  }\CommentTok{\# Redondear hacia arriba}
\NormalTok{\}}

\CommentTok{\# Parámetros del problema}
\NormalTok{sigma }\OtherTok{\textless{}{-}} \DecValTok{10}
\NormalTok{B }\OtherTok{\textless{}{-}} \DecValTok{2}

\CommentTok{\# Cálculos para diferentes niveles de confianza}
\FunctionTok{cat}\NormalTok{(}\StringTok{"Tamaño de muestra requerido:}\SpecialCharTok{\textbackslash{}n}\StringTok{"}\NormalTok{)}
\end{Highlighting}
\end{Shaded}

\begin{verbatim}
## Tamaño de muestra requerido:
\end{verbatim}

\begin{Shaded}
\begin{Highlighting}[]
\FunctionTok{cat}\NormalTok{(}\StringTok{"90\% confianza:"}\NormalTok{, }\FunctionTok{calcular\_n}\NormalTok{(sigma, B, }\FloatTok{0.90}\NormalTok{), }\StringTok{"cazadores}\SpecialCharTok{\textbackslash{}n}\StringTok{"}\NormalTok{)}
\end{Highlighting}
\end{Shaded}

\begin{verbatim}
## 90% confianza: 68 cazadores
\end{verbatim}

\begin{Shaded}
\begin{Highlighting}[]
\FunctionTok{cat}\NormalTok{(}\StringTok{"95\% confianza:"}\NormalTok{, }\FunctionTok{calcular\_n}\NormalTok{(sigma, B, }\FloatTok{0.95}\NormalTok{), }\StringTok{"cazadores}\SpecialCharTok{\textbackslash{}n}\StringTok{"}\NormalTok{)}
\end{Highlighting}
\end{Shaded}

\begin{verbatim}
## 95% confianza: 97 cazadores
\end{verbatim}

\begin{Shaded}
\begin{Highlighting}[]
\FunctionTok{cat}\NormalTok{(}\StringTok{"99\% confianza:"}\NormalTok{, }\FunctionTok{calcular\_n}\NormalTok{(sigma, B, }\FloatTok{0.99}\NormalTok{), }\StringTok{"cazadores}\SpecialCharTok{\textbackslash{}n}\StringTok{"}\NormalTok{)}
\end{Highlighting}
\end{Shaded}

\begin{verbatim}
## 99% confianza: 166 cazadores
\end{verbatim}

\#\#\#Problema 5

La Agencia de Protección Ambiental ha recolectado datos sobre mediciones
de \(LC50\) (concentraciones que matan a \(50 \%\) de los animales de
prueba) para ciertos productos químicos que es probable se encuentren en
ríos y lagos de agua dulce. Para cierta especie de peces, las mediciones
de LC50 (en partes por millón) de DDT en 12 experimentos fueron las
siguientes:

\[16, 5, 21, 19, 10, 5, 8, 2, 7, 2, 4, 9\]

Calcule la media real de LC50 para DDT con un coeficiente de confianza
\(0.90\). Suponga que las mediciones de LC50 tienen una distribución
aproximadamente normal.

\textbf{Respuesta:}

\begin{Shaded}
\begin{Highlighting}[]
\CommentTok{\# Datos}
\NormalTok{datos }\OtherTok{\textless{}{-}} \FunctionTok{c}\NormalTok{(}\DecValTok{16}\NormalTok{, }\DecValTok{5}\NormalTok{, }\DecValTok{21}\NormalTok{, }\DecValTok{19}\NormalTok{, }\DecValTok{10}\NormalTok{, }\DecValTok{5}\NormalTok{, }\DecValTok{8}\NormalTok{, }\DecValTok{2}\NormalTok{, }\DecValTok{7}\NormalTok{, }\DecValTok{2}\NormalTok{, }\DecValTok{4}\NormalTok{, }\DecValTok{9}\NormalTok{)}

\CommentTok{\# Parámetros}
\NormalTok{n }\OtherTok{\textless{}{-}} \FunctionTok{length}\NormalTok{(datos)}
\NormalTok{confianza }\OtherTok{\textless{}{-}} \FloatTok{0.90}
\NormalTok{alpha }\OtherTok{\textless{}{-}} \DecValTok{1} \SpecialCharTok{{-}}\NormalTok{ confianza}
\CommentTok{\# Realizar prueba t (solo para obtener el IC)}
\NormalTok{resultado }\OtherTok{\textless{}{-}} \FunctionTok{t.test}\NormalTok{(datos, }\AttributeTok{conf.level =}\NormalTok{ confianza)}

\CommentTok{\# Mostrar resultados}
\FunctionTok{print}\NormalTok{(resultado)}
\end{Highlighting}
\end{Shaded}

\begin{verbatim}
## 
##  One Sample t-test
## 
## data:  datos
## t = 4.8529, df = 11, p-value = 0.0005084
## alternative hypothesis: true mean is not equal to 0
## 90 percent confidence interval:
##   5.669423 12.330577
## sample estimates:
## mean of x 
##         9
\end{verbatim}

\begin{Shaded}
\begin{Highlighting}[]
\FunctionTok{cat}\NormalTok{(}\StringTok{"}\SpecialCharTok{\textbackslash{}n}\StringTok{Intervalo de confianza al "}\NormalTok{, confianza}\SpecialCharTok{*}\DecValTok{100}\NormalTok{, }\StringTok{"\%:}\SpecialCharTok{\textbackslash{}n}\StringTok{"}\NormalTok{, }\AttributeTok{sep =} \StringTok{""}\NormalTok{)}
\end{Highlighting}
\end{Shaded}

\begin{verbatim}
## 
## Intervalo de confianza al 90%:
\end{verbatim}

\begin{Shaded}
\begin{Highlighting}[]
\FunctionTok{cat}\NormalTok{(}\StringTok{"IC(μ) = ["}\NormalTok{, }\FunctionTok{round}\NormalTok{(resultado}\SpecialCharTok{$}\NormalTok{conf.int[}\DecValTok{1}\NormalTok{], }\DecValTok{3}\NormalTok{), }\StringTok{", "}\NormalTok{, }
    \FunctionTok{round}\NormalTok{(resultado}\SpecialCharTok{$}\NormalTok{conf.int[}\DecValTok{2}\NormalTok{], }\DecValTok{3}\NormalTok{), }\StringTok{"] ppm}\SpecialCharTok{\textbackslash{}n}\StringTok{"}\NormalTok{, }\AttributeTok{sep =} \StringTok{""}\NormalTok{)}
\end{Highlighting}
\end{Shaded}

\begin{verbatim}
## IC(μ) = [5.669, 12.331] ppm
\end{verbatim}

\#\#\#Problema 6

Se encontrará que, con probabilidad en la cercanía de \(0.95\), muchas
variables aleatorias observadas en la naturaleza se encuentran a no más
de \(2\) desviaciones estándar de sus medias

\begin{enumerate}
\def\labelenumi{\alph{enumi})}
\tightlist
\item
  Calcule la \(P(\mu − 2\sigma ≤ Y ≤ \mu + 2\sigma)\), donde Y tiene una
  distribución Normal con media \(\mu\) y varianza \(\sigma^2\).
\end{enumerate}

\begin{Shaded}
\begin{Highlighting}[]
\CommentTok{\# Para Normal, estandarizamos: Z = (Y {-} μ)/σ \textasciitilde{} N(0,1)}
\CommentTok{\# Entonces P(μ{-}2σ \textless{}= Y \textless{}= μ+2σ) = P({-}2 \textless{}= Z \textless{}= 2) = Φ(2) {-} Φ({-}2)}
\NormalTok{prob\_normal }\OtherTok{\textless{}{-}} \FunctionTok{pnorm}\NormalTok{(}\DecValTok{2}\NormalTok{) }\SpecialCharTok{{-}} \FunctionTok{pnorm}\NormalTok{(}\SpecialCharTok{{-}}\DecValTok{2}\NormalTok{)}
\NormalTok{prob\_normal}
\end{Highlighting}
\end{Shaded}

\begin{verbatim}
## [1] 0.9544997
\end{verbatim}

\begin{enumerate}
\def\labelenumi{\alph{enumi})}
\setcounter{enumi}{1}
\tightlist
\item
  Calcule la \(P(\mu − 2\sigma ≤ Y ≤ \mu + 2\sigma)\), donde Y tiene una
  distribución Uniforme en el intervalo \((a,b)\).
\end{enumerate}

\begin{Shaded}
\begin{Highlighting}[]
\CommentTok{\# Para Uniforme(a,b): μ=(a+b)/2 y σ=(b{-}a)/sqrt(12)}
\CommentTok{\# El intervalo μ ± 2σ tiene semiamplitud (b{-}a)/√3, que es mayor que (b{-}a)/2.}
\CommentTok{\# Por lo tanto, cubre completamente [a,b] y la probabilidad es 1.}
\NormalTok{prob\_uniform }\OtherTok{\textless{}{-}} \DecValTok{1}
\NormalTok{prob\_uniform}
\end{Highlighting}
\end{Shaded}

\begin{verbatim}
## [1] 1
\end{verbatim}

\begin{enumerate}
\def\labelenumi{\alph{enumi})}
\setcounter{enumi}{2}
\tightlist
\item
  Calcule la \(P(\mu − 2\sigma ≤ Y ≤ \mu + 2\sigma)\), donde Y tiene una
  distribución exponcial con parámetro \(\beta\).
\end{enumerate}

\begin{Shaded}
\begin{Highlighting}[]
\CommentTok{\# Convención: Usamos parámetro β como *escala*: μ=β, σ=β.}
\CommentTok{\# Entonces μ {-} 2σ = {-}β (truncado en 0 porque Y\textgreater{}=0) y μ + 2σ = 3β.}
\CommentTok{\# P = P(0 \textless{}= Y \textless{}= 3β) = 1 {-} exp({-}3β/β) = 1 {-} e\^{}\{{-}3\}.}
\NormalTok{prob\_exp }\OtherTok{\textless{}{-}} \DecValTok{1} \SpecialCharTok{{-}} \FunctionTok{exp}\NormalTok{(}\SpecialCharTok{{-}}\DecValTok{3}\NormalTok{)}
\NormalTok{prob\_exp}
\end{Highlighting}
\end{Shaded}

\begin{verbatim}
## [1] 0.9502129
\end{verbatim}

\begin{enumerate}
\def\labelenumi{\alph{enumi})}
\setcounter{enumi}{3}
\tightlist
\item
  Los incisos anteriores, cumplen la desigualdad de Chebyshef. Explique
\end{enumerate}

\begin{Shaded}
\begin{Highlighting}[]
\NormalTok{k }\OtherTok{\textless{}{-}} \DecValTok{2}
\NormalTok{chebyshev\_bound }\OtherTok{\textless{}{-}} \DecValTok{1} \SpecialCharTok{{-}} \DecValTok{1}\SpecialCharTok{/}\NormalTok{k}\SpecialCharTok{\^{}}\DecValTok{2}  \CommentTok{\# 1 {-} 1/4 = 3/4}
\NormalTok{cota\_chebyshev }\OtherTok{\textless{}{-}}\NormalTok{ chebyshev\_bound}

\NormalTok{resumen }\OtherTok{\textless{}{-}} \FunctionTok{data.frame}\NormalTok{(}
  \AttributeTok{Distribucion =} \FunctionTok{c}\NormalTok{(}\StringTok{"Normal(μ,σ\^{}2)"}\NormalTok{, }\StringTok{"Uniforme(a,b)"}\NormalTok{, }\StringTok{"Exponencial(β, escala)"}\NormalTok{),}
  \AttributeTok{Prob\_dentro\_2sigma =} \FunctionTok{c}\NormalTok{(prob\_normal, prob\_uniform, prob\_exp),}
  \AttributeTok{Cota\_Chebyshev\_k2 =} \FunctionTok{rep}\NormalTok{(cota\_chebyshev, }\DecValTok{3}\NormalTok{)}
\NormalTok{)}
\NormalTok{resumen}
\end{Highlighting}
\end{Shaded}

\begin{verbatim}
##             Distribucion Prob_dentro_2sigma Cota_Chebyshev_k2
## 1          Normal(μ,σ^2)          0.9544997              0.75
## 2          Uniforme(a,b)          1.0000000              0.75
## 3 Exponencial(β, escala)          0.9502129              0.75
\end{verbatim}

La desigualdad de Chebyshev dice, para cualquier v.a. con media y
varianza finitas, que \[P(|Y-\mu| \leq k\sigma) \geq 1 - \frac{1}{k^2}\]

.

\end{document}
